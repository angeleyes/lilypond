% lilyponddefs.tex
%
% Include file for LilyPond.
%
% This file defines various macros to acommodate lilypond output.
%
% It should run with plain TeX, LaTeX, pdftex, and texinfo.
%
% To avoid interferences, lilyponddefs.tex should be loaded within a group.
% To load it only once, most of the definitions must be global.
%
% The overall structure of a file created by LilyPond is as follows:
%
%   <lilypond parameter definitions>
%   \ifx\lilypondstart \undefined
%     \input lilyponddefs
%   \fi
%   \lilypondstart
%   <font setup and note output>
%   \lilypondend
%
% No footers and headers are provided for the stand-alone run (i.e., for
% directly saying `latex <LilyPond output>'.
%
%
% Avoid \par while reading this file.
\edef\lilyponddefsELC{\the\endlinechar}%
\endlinechar -1\relax

% the next three macros are taken from LaTeX
\long\gdef\lilypondfirst#1#2{#1}

\long\gdef\lilypondsecond#1#2{#2}

\gdef\lilypondifundefined#1{
  \expandafter\ifx\csname#1\endcsname\relax
    \expandafter\lilypondfirst
  \else
    \expandafter\lilypondsecond
  \fi
}

\gdef\lilypondstart{
  \begingroup
  \catcode `\@=11\relax
  % \@nodocument is defined as \relax after `\begin{document}'
  \lilypondifundefined{@nodocument}
    {
      % either plain TeX or texinfo or not at the beginning of LaTeX input
      \def\x{\endgroup}
    }
    {
      % provide a proper LaTeX preamble (for A4 paper format)
      \def\x{
        \endgroup
        \def\lilyponddocument{}
        \documentclass[a4paper]{article}
        \pagestyle{empty}
        % \begin is defined as \outer in texinfo, thus we use \csname
        \csname begin\endcsname{document}
        % center staves horizontally on page
        \ifdim\lilypondpaperlinewidth\lilypondpaperunit > 0pt
          \hsize\lilypondpaperlinewidth\lilypondpaperunit
          % we abuse \scoreshift temporarily
          \scoreshift \paperwidth
          \advance\scoreshift -\the\hsize
          \scoreshift 0.5\scoreshift
          \advance\scoreshift -1in
          \oddsidemargin \scoreshift
          \evensidemargin \scoreshift
        \fi
        \parindent 0pt
      }
    }

  \x

  \lilypondifundefined{mustmakelilypondtitle}
    {}
    {\makelilypondtitle}

  \lilypondifundefined{mustmakelilypondpiecetitle}
    {}
    {\makelilypondpiecetitle}
}

\gdef\lilypondend{
  \lilypondifundefined{lilypondbook}
    {\lilypondifundefined{lilypondpaperlastpagefill}
      {\vskip 0pt plus \lilypondpaperinterscorelinefill00 fill}
      {}
    }
    {}

  \begingroup
  \lilypondifundefined{lilyponddocument}
    {
      \def\x{\endgroup}
    }
    {
      \def\x{
        \endgroup
        \csname end\endcsname{document}
      }
    }

  \x
}

% this is an inversed \loop ... \repeat macro
\def\lilypondloop#1\lilypondrepeat{
  \def\lilypondbody{#1}
  \lilyponditerate
}

\def\lilyponditerate{
  % \if ...
    \lilypondbody
    \let\lilypondnext \relax
  \else
    \let\lilypondnext \lilyponditerate
  \fi
  \lilypondnext
}

\newread\lilypondinput

% the following macro is executed only once
\gdef\lilypondspecial{
  \special{header=music-drawing-routines.ps}
  \gdef\lilypondspecial{}
}

% the feta characters
\input feta20

\global\font\fetasixteen = feta16
\gdef\fetafont{\fetasixteen}
\gdef\fetachar#1{\hbox{\fetasixteen#1}}

\gdef\botalign#1{
  \vbox to 0pt{\vss #1}
}
\gdef\leftalign#1{
  \hbox to 0pt{#1\hss}
}

\gdef\lyitem#1#2#3{
  \botalign{
    \hbox{\raise #1\outputscale
          \leftalign{\kern #2\outputscale #3}}
  }
}

\gdef\lybox#1#2#3{
  \hbox to #1\outputscale {
    \lower\scoreshift \vbox to #2\outputscale {
      \hbox{#3}
      \vss
    }
    \hss
  }
}

\gdef\lyvrule#1#2#3#4{
  \kern #1\outputscale
  \vrule width #2\outputscale depth #3\outputscale height #4\outputscale
}

% Attempt to keep lilypiecetitle together with the piece:
%
% TODO: figure this out.
\gdef\myfilbreak{}%\par\vfil\penalty200\vfilneg}

\lilypondifundefined{lilypondpaperinterscorelinefill}
  {\gdef\lilypondpaperinterscorelinefill{0}}
  {\gdef\lilypondpaperinterscorelinefill{1}}

\gdef\interscoreline{
  \vskip \lilypondpaperinterscoreline \lilypondpaperunit
    plus \lilypondpaperinterscorelinefill fill
}

% Are we using PDFTeX?  If so, use pdf definitions.
% MiKTeX checks \pdfoutput the wrong way, thus we use \csname.
\lilypondifundefined{lilypondpostscript}
  {
    \lilypondifundefined{pdfoutput}
      {\input lily-ps-defs }
      {
        \pdfoutput = 1
        \input lily-pdf-defs
      }
  }
  {}

\newdimen\outputscale
\newdimen\scoreshift

% Restore newline functionality (disabled to avoid \par).
\endlinechar \lilyponddefsELC
\endinput

% EOF
