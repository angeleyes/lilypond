\documentclass{article}
\def\kdots{,\ldots,}
\title{Not the Font-En-Tja font}
\author{HWN \& JCN} 
\begin{document}
\maketitle


\section{Introduction}

This document are some design notes of the Feta font.  Feta (not an
abbreviation of Font-En-Tja) is a font of music symbols.  All MetaFont
%ugh sources are original.  The symbols are modelled after various
editions of music, notably
\begin{itemize}
\item B\"arenreiter
\item Hofmeister
\item Breitkopf
\item Durand \& C'ie
\end{itemize}

The best references on Music engraving are Wanske\cite{wanske} and
Ross\cite{ross} quite some of their insights were used.  Although it
is a matter of taste, I'd say that B\"arenreiter has the finest
typography of all.


\section{Bezier curves for slurs}

Objective:  slurs in music are curved objects designating that notes
should fluently bound.  They are drawn as smooth curves, with their
center thicker and the endings tapered.

There are some variants: the simplest slur shape only has the width as
parameter.  Then we give some suggestions for tuning the shapes.  The
simple slur algorithm is used for drawing ties as well.



\subsection{Simple slurs}

Long slurs are flat, whereas short slurs look like small circle arcs.
Details are given in Wanske\cite{ross} and Ross\cite{wanske}.  The
shape of a slur can be given as a Bezier curve with four control
points:

\begin{eqnarray*}
  B(t) &=& (1-t)^3c_1 +3(1-t)^2tc_2 + 3(1-t)t^2c_3 + t^3c_4.
\end{eqnarray*}

We will assume that the slur connects two notes of the same
pitch.  Different slurs can be created by rotating the derived shape.
We will also assume that the slur has a vertical axis of symmetry
through its center.  The left point will be the origin.     So we have
the following equations for the control points $c_1\kdots c_4$.

\begin{eqnarray*}
c_1&=& (0,0)\\
c_2&=& (i, h)\\
c_3&=& (b-i, h)\\
c_4&=& (b, 0)
\end{eqnarray*}

The quantity $b$ is given, it is the width of the slur.  The
conditions on the shape of the slur for small and large $b$ transform
to
\begin{eqnarray*}
 h \to h_{\infty} , &&\quad b \to \infty\\
 h \approx r_{0} b, &&\quad b \to 0.
\end{eqnarray*}
To tackle  this, we  will  assume that $h   = F(b)$, for  some kind of
$F(\cdot)$.  One function that satisfies the above conditions is
$$
F(b) = h_{\infty} \frac{2}{\pi} \arctan \left( \frac{\pi r_0}{2
h_{\infty}} b \right).
$$

For satisfying results we choose $h_{\infty} = 2\cdot \texttt{interline}$
and $r_0 = \frac 13$.

\subsection{Height correction}

Aside from being a smooth curve, slurs should avoid crossing
enclosed notes and their stems.

An easy way to achieve this is to extend the slur's height,
so that the slur will curve just above any disturbing notes.

The parameter $i$ determines the flatness of the curve.  Satisfying
results have been obtained with $i = h$.

The formula can be generalised to allow for corrections in the shape, 
\begin{eqnarray*}
c_1&=& (0,0)\\
c_2&=& (i', h')\\
c_3&=& (b-i', h')\\
c_4&=& (b, 0)
\end{eqnarray*}
Where
$$
i' = h(b) (1 + i_{corr}), \quad h' = h(b) (1 + h_{corr}).
$$

The default values for these corrections are $0$.  A $h_{corr}$ that is
negative, makes the curve flatter in the center.  A $h_{corr}$ that is
positive make the curve higher. 

At every encompassed note's x position the difference $\delta _y$ 
between the slur's height and the note is calculated.  The greatest 
$\delta _y$ is used to calculate $h_{corr}$ is by lineair extrapolation.

However, this simple method produces satisfactory results only for 
small and symmetric disturbances.


\subsection{Tangent method correction}

A somewhat more elaborate\footnote{While staying in the realm 
of emperic computer science} way of having a slur avoid 
disturbing notes is by first defining the slur's ideal shape 
and then using the height correction.  The ideal shape of a 
slur can be guessed by calculating the tangents of the disturbing 
notes:
% a picture wouldn't hurt...
\begin{eqnarray*}
  y_{disturb,l} &=& \rm{rc}_l x\\
  y_{disturb,r} &=& \rm{rc}_r + c_{3,x},
\end{eqnarray*}
where
\begin{eqnarray*}
  \rm{rc}_l &=& \frac{y_{disturb,l} - y_{encompass,1}}
    {x_{disturb,l} - x_{encompass,1}}\dot x\\
  \rm{rc}_r &=& \frac{y_{encompass,n} - y_{disturb,r}}
    {x_{encompass,n} - x_{disturb,r}} \dot x + c_{3,x}.
\end{eqnarray*}

We assume that having the control points $c_2$ and $c_3$ located 
on tangent$_1$ and tangent$_2$ resp. 
% t: tangent
\begin{eqnarray*}
  y_{tangent,l} &=& \alpha \rm{rc}_l x\\
  y_{tangent,r} &=& \alpha \rm{rc}_r + c_{3,x}.
\end{eqnarray*}

Beautiful slurs have rather strong curvature at the extreme
control points.  That's why we'll have $\alpha > 1$.
Satisfactory resulsts have been obtained with
$$
  \alpha \approx 2.4.
$$

The positions of control points $c_2$ and $c_3$ are obtained
by solving with the height-line
\begin{eqnarray*}
  y_h &=& \rm{rc}_h + c_h.
\end{eqnarray*}

The top-line runs through the points disturb$_{left}$ and
disturb$_{right}$.  In the case that 
$$
z_{disturb,l} = z_{disturb,r},
$$
we'll have 
$$
  \angle(y_{tangent,l},y_h) = \angle(y_{tangent,r},y_h).
$$



\section{Sizes}

Traditional engraving uses a set of 9 standardised sizes for Staffs
(running from 0 to 8).  

We have tried to measure these (helped by a magnifying glass), and
found the staffsizes in the following table.  One should note that
these are estimates, so I think there could be a measuring error of ~
.5 pt.  Moreover [Ross] states that not all engravers use exactly
those sizes.

\begin{table}
\begin{tabular}{lll}
Staffsize	&Numbers		&Name\\
\hline\\
26.2pt	&No. 0\\
22.6pt	&No. 1		&Giant/English\\
21.3pt	&No. 2		&Giant/English\\
19.9pt	&No. 3		&Regular, Ordinary, Common\\
19.1pt	&No. 4		&Peter\\
17.1pt	&No. 5		&Large middle\\
15.9pt	&No. 6		&Small middle\\
13.7pt	&No. 7		&Cadenza\\
11.1pt	&No. 8		&Pearl\\
\end{tabular}
\caption{Font and staff sizes}
\end{table}


This table is partially taken from [Ross].  Most music is set in No.3,
but the papersizes usually are bigger than standard printer paper
(such as A4).  If you plot these, you'll notice that the sizes (With
exception of 26) almost (but not quite) form a arithmetic progression.

Ross states that the dies (the stamps to make the symbols) come in
12 different sizes.

\bibliographystyle{plain}
\bibliography{engraving}



\end{document}
